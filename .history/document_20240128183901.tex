\documentclass[a4paper,12pt]{article}
\usepackage[UTF8]{ctex}
\usepackage{tocloft}
\usepackage{graphicx}
% 重定义目录命令,以居中显示目录标题
\renewcommand{\contentsname}{\hfill\bfseries\Large 目录\hfill}   
\renewcommand{\cftaftertoctitle}{\hfill}

\begin{document}
% 更改封面的字体样式
\title{\huge My Test Document}
\author{{\large knight-zzm}}
\date{\today}
\maketitle
\thispagestyle{empty} % 封面页无页码
\newpage

\tableofcontents
\thispagestyle{empty} % 无页码
\newpage
\pagenumbering{arabic}
\section{练习latex 论文写作}
This is the introduction.

\subsection{图表的创建和引用}

    \begin{figure}[h]
        \centering
        \includegraphics[width=1\textwidth]{./Figure~\ref{fig:}/时频图和PBC分布图.png}
        \caption{时频图和PBC分布图}
        \label{fig:time-freq}
    \end{figure}

\subsection{公式编辑和引用}

\subsection{表格编辑和引用}
    \begin{tabular}{|c|c|c|c|}
        \hline
        \multicolumn{2}{|c|}{MCM} & \multicolumn{2}{|c|}{ICM} \\
        \hline
        A & 连续型 & D & 运筹学/网络科学\\
        \hline
        B & 离散型 & E & 环境科学\\
        \hline
        C & 大数据 & F & 政策\\
        \hline
    \end{tabular}
    
    \begin{tabular}{rc}
    Apples & Green\\
    \hline 
    Strawberries & Red \\
    \cline{1-1}
    Oranges & Orange \\
    \end{tabular}
    
    \begin{tabular}{|r|l|}
    \hline
    8 & here's \\
    \cline{2-2}
    86 & stuff\\
    \hline \hline 
    2008 & now \\
    \hline 
    \end{tabular}

\end{document}

